\documentclass[12pt]{article}
\usepackage{graphicx} % Required for inserting images
\usepackage[utf8]{inputenc}

\usepackage{hyperref}
\usepackage{xcolor}

\hypersetup{
  colorlinks=true,
  citecolor=blue,
  linkcolor=red,
  urlcolor=magenta,
  }
  % citebordercolor=false}

\usepackage{float}
\usepackage{url}
\usepackage{tabularx}
\usepackage{amsmath}
\usepackage{amsfonts}
\usepackage{amssymb}
\usepackage{natbib}
\usepackage{xurl}
\usepackage{lineno}
\usepackage{amsthm}
\usepackage{epsfig}
\usepackage[linesnumbered, ruled]{algorithm2e}

\usepackage{ulem}
\newcommand{\stkout}[1]{\ifmmode\text{\sout{\ensuremath{#1}}}\else\sout{#1}\fi}

\usepackage{multirow}
\usepackage{longtable}
\setlength\LTleft{0pt}
\usepackage{tabularx}

\usepackage{graphicx}
\usepackage{xcolor}
\usepackage{float}
\usepackage{url}
% \setlength\LTleft{0pt}
\usepackage{tabularx}
\usepackage{amsmath}
\usepackage{amsfonts}
\usepackage{amssymb}
\usepackage{natbib}
% \bibliographystyle{unsrt}
\bibliographystyle{agu}
% \bibliographystyle{plainnat}

\newcommand\prob{\mathbb{P}}

\usepackage{xurl} %
%
\usepackage{lineno}

\usepackage{soul}
%
%
\usepackage{amsmath}
\usepackage{amssymb}
\usepackage{amsthm}
\usepackage{float}
%
\usepackage{epsfig}
\usepackage[linesnumbered, ruled]{algorithm2e}

\usepackage{natbib}
\usepackage{color}

%
%
%

\usepackage{multirow}
\usepackage{longtable}
\setlength\LTleft{0pt}
\usepackage{tabularx}
%
\clearpage{}%


\newcommand{\xh}[1]{\textcolor{orange}{\textbf{(xh:)} #1}}

\newcommand{\aj}[1]{\textcolor{magenta}{\textbf{(aj:)} #1}}

%
\newcommand{\resp}[1]{\textcolor{red}{\textbf{Response: } #1}}
\newcommand{\respc}[1]{\textcolor{red}{#1}}


%
\renewcommand{\[}{\left[}
\renewcommand{\]}{\right]}
\renewcommand{\(}{\left(}
\renewcommand{\)}{\right)}

%
\newcommand{\dd}[2]{\frac{d #1}{d #2}}
\newcommand{\ddt}[1]{\frac{d #1}{d t}}
\newcommand{\ddd}[2]{\frac{d^2 #1}{d #2^2}}
\newcommand{\dddt}[1]{\frac{d^2 #1}{d t^2}}
\newcommand{\pp}[2]{\frac{\partial #1}{\partial #2}}
\newcommand{\ppp}[2]{\frac{\partial^2 #1}{\partial #2^2}}
\newcommand{\pppp}[2]{\frac{\partial^3 #1}{\partial #2^3}}
\newcommand{\ppppp}[2]{\frac{\partial^4 #1}{\partial #2^4}}

\newcommand{\adj}[1]{#1^{*}}
\newcommand{\abs}[1]{\left|#1\right|}
\newcommand{\divergence}[1]{\nabla \cdot #1}
\newcommand{\enorm}[1]{\vvvert #1 \vvvert}
\newcommand{\grad}[1]{\nabla #1}
\newcommand{\laplace}[1]{\nabla^2 #1}
\newcommand{\norm}[2]{\left\|\, #1 \,\right\|_{#2}}
\newcommand{\order}[1]{\mathcal{O}\(#1\)}
\newcommand{\supp}{\mathop{\mathrm{supp}}}
\newcommand{\vvvert}{|\kern-1pt|\kern-1pt|}

\newcommand{\eq}[1]{\mathop{\,{\buildrel #1 \over =}\,}}
\newcommand{\ap}[1]{\mathop{\,{\buildrel #1 \over \approx}\,}}

%

%
\newcommand{\hg}{\hat{g}}
\newcommand{\hh}{\hat{h}}
\newcommand{\hi}{\hat{i}}
\newcommand{\hj}{\hat{j}}
\newcommand{\hk}{\hat{k}}
\newcommand{\hm}{\hat{m}}
\newcommand{\hn}{\hat{n}}
\newcommand{\hs}{\hat{s}}
\newcommand{\hu}{\hat{u}}
\newcommand{\hv}{\hat{v}}
\newcommand{\hx}{\hat{x}}

\newcommand{\hA}{\hat{A}}
\newcommand{\hC}{\hat{C}}
\newcommand{\hI}{\hat{I}}
\newcommand{\hJ}{\hat{J}}
\newcommand{\hN}{\hat{N}}
\newcommand{\hT}{\hat{T}}
\newcommand{\hU}{\hat{U}}

\newcommand{\hbf}{\boldsymbol{\hat{f}}}
\newcommand{\hbg}{\boldsymbol{\hat{g}}}
\newcommand{\hsig}{\hat{\sigma}}

%
\newcommand{\barg}{\bar{g}}
\newcommand{\barh}{\bar{h}}
\newcommand{\barm}{\bar{m}}
\newcommand{\barv}{\bar{v}}
\newcommand{\barw}{\bar{w}}
\newcommand{\barx}{\bar{x}}
\newcommand{\bary}{\bar{y}}

\newcommand{\barB}{\bar{B}}
\newcommand{\barC}{\bar{C}}
\newcommand{\barH}{\bar{H}}
\newcommand{\barJ}{\bar{J}}
\newcommand{\barN}{\bar{N}}
\newcommand{\barR}{\bar{R}}

\newcommand{\barmu}{\bar{\mu}}

%
\newcommand{\tf}{\tilde{f}}
\newcommand{\thh}{\tilde{h}}

\newcommand{\tA}{\tilde{A}}
\newcommand{\tg}{\tilde{g}}
\newcommand{\tH}{\tilde{H}}
\newcommand{\tJ}{\tilde{J}}
\newcommand{\tQ}{\tilde{Q}}
\newcommand{\tT}{\tilde{T}}

\newcommand{\tmu}{\tilde{\mu}}
\newcommand{\ttheta}{\tilde{\theta}}
\newcommand{\txi}{\tilde{\xi}}

\newcommand{\tTheta}{\tilde{\Theta}}
\newcommand{\tXi}{\tilde{\Xi}}

%
\newcommand{\mb}[1]{\mathbf{#1}}
\newcommand{\sbf}[1]{\boldsymbol{#1}}

\newcommand{\bb}{\textbf{b}}
\newcommand{\bd}{\textbf{d}}
\newcommand{\bee}{\textbf{e}}
\newcommand{\bff}{\textbf{f}}
\newcommand{\bh}{\textbf{h}}
\newcommand{\bg}{\textbf{g}}
\newcommand{\bk}{\textbf{k}}
\newcommand{\bii}{\textbf{i}}
\newcommand{\bj}{\textbf{j}}
\newcommand{\bl}{\textbf{l}}
\newcommand{\bn}{\textbf{n}}
\newcommand{\bp}{\textbf{p}}
\newcommand{\br}{\textbf{r}}
\newcommand{\bs}{\textbf{s}}
\newcommand{\bt}{\textbf{t}}
\newcommand{\bu}{\textbf{u}}
\newcommand{\bv}{\textbf{v}}
\newcommand{\bw}{\textbf{w}}
\newcommand{\bx}{\textbf{x}}
\newcommand{\by}{\textbf{y}}

\newcommand{\bA}{\mathbf{A}}
\newcommand{\bC}{\mathbf{C}}
\newcommand{\bE}{\mathbf{E}}
\newcommand{\bF}{\mathbf{F}}
\newcommand{\bG}{\mathbf{G}}
\newcommand{\bI}{\mathbf{I}}
\newcommand{\bK}{\mathbf{K}}
\newcommand{\bN}{\mathbf{N}}
\newcommand{\bQ}{\mathbf{Q}}
\newcommand{\bR}{\mathbf{R}}
\newcommand{\bT}{\mathbf{T}}
\newcommand{\bU}{\mathbf{U}}
\newcommand{\bV}{\mathbf{V}}
\newcommand{\bY}{\mathbf{Y}}

\newcommand{\balpha}{\boldsymbol{\alpha}}
\newcommand{\bbeta}{\boldsymbol{\beta}}
\newcommand{\bepsilon}{\boldsymbol{\epsilon}}
\newcommand{\bhsig}{\boldsymbol{\hsig}}
\newcommand{\bpsi}{\boldsymbol{\psi}}
\newcommand{\bsig}{\boldsymbol{\sigma}}
\newcommand{\btau}{\boldsymbol{\tau}}
\newcommand{\bmu}{\boldsymbol{\mu}}
\newcommand{\btheta}{\boldsymbol{\theta}}
\newcommand{\bphi}{\boldsymbol{\phi}}
\newcommand{\bxi}{\boldsymbol{\xi}}

\newcommand{\bDelta}{\boldsymbol{\Delta}}
\newcommand{\bTheta}{\boldsymbol{\Theta}}
\newcommand{\bXi}{\boldsymbol{\Xi}}
\newcommand{\bOmega}{\boldsymbol{\Omega}}
\newcommand{\bSigma}{\boldsymbol{\Sigma}}

%
\newcommand{\EE}{\mathbb{E}}
\newcommand{\II}{\mathbb{I}}
\newcommand{\NN}{\mathbb{N}}
\newcommand{\QQ}{\mathbb{Q}}
\newcommand{\PP}{\mathbb{P}}
\newcommand{\RR}{\mathbb{R}}

%
\newcommand{\vt}{\vec{t}}
\newcommand{\vu}{\vec{u}}
\newcommand{\vv}{\vec{v}}
\newcommand{\vx}{\vec{x}}

\newcommand{\vV}{\vec{V}}

\newcommand{\vo}{\vec{\omega}}

%
\newcommand{\dotk}{\dot{k}}
\newcommand{\dotm}{\dot{m}}
\newcommand{\dotx}{\dot{x}}

\newcommand{\dotomega}{\dot{\omega}}

%
\newcommand{\CA}{\mathcal{A}}
\newcommand{\CB}{\mathcal{B}}
\newcommand{\CD}{\mathcal{D}}
\newcommand{\CE}{\mathcal{E}}
\newcommand{\CF}{\mathcal{F}}
\newcommand{\CH}{\mathcal{H}}
\newcommand{\CJ}{\mathcal{J}}
\newcommand{\CK}{\mathcal{K}}
\newcommand{\CL}{\mathcal{L}}
\newcommand{\CM}{\mathcal{M}}
\newcommand{\CN}{\mathcal{N}}
\newcommand{\CR}{\mathcal{R}}
\newcommand{\CS}{\mathcal{S}}
\newcommand{\CT}{\mathcal{T}}
\newcommand{\CU}{\mathcal{U}}
\newcommand{\CV}{\mathcal{V}}
\newcommand{\CW}{\mathcal{W}}
\newcommand{\CX}{\mathcal{X}}
\newcommand{\CY}{\mathcal{Y}}

\newcommand{\CbarJ}{\bar{\mathcal{J}}}
\newcommand{\CbarL}{\bar{\mathcal{L}}}
\newcommand{\CbarR}{\bar{\mathcal{R}}}

%
\newcommand{\du}{\delta{u}}

\newcommand{\dbeta}{\delta{\beta}}
\newcommand{\dxi}{\delta{\xi}}
\newcommand{\deta}{\delta{\eta}}
\newcommand{\drho}{\delta{\rho}}
\newcommand{\dtau}{\delta{\tau}}

\newcommand{\dbu}{\delta{\boldsymbol{u}}}
\newcommand{\dbp}{\delta{\boldsymbol{p}}}
\newcommand{\dbx}{\delta{\boldsymbol{x}}}

\newcommand{\Dx}{\Delta{x}}
\newcommand{\Dy}{\Delta{y}}
\newcommand{\Dt}{\Delta{t}}

%
\newcommand{\myblue}[1]{{\color[rgb]{0,0,0.65} #1}}
\newcommand{\mygreen}[1]{{\color[rgb]{0,.65,0} #1}}
\newcommand{\mywhite}[1]{{\color[rgb]{1.0,1.0,1.0} #1}}
\newcommand{\myred}[1]{{\color[rgb]{0.65,0.0,0.0} #1}}
\newcommand{\myblack}[1]{{\color[rgb]{0.0,0.0,0.0} #1}}
\newcommand{\mygrey}[1]{{\color[rgb]{0.6,0.6,0.6} #1}}

%
% \newcommand{\coo}{CO$_2$}
% \newcommand{\hho}{H$_2$O}
% \newcommand{\oo}{O$_2$}
% \newcommand{\nn}{N$_2$}
% \newcommand{\mwe}{MW$_e$}
% \newcommand{\nox}{NO$_{\textrm{x}}$}
% \newcommand{\cooe}{CO$_{2e}$}
% \newcommand{\nno}{N$_2$O}
% \newcommand{\noo}{NO$_2$ }
% \newcommand{\chhhh}{CH$_4$ }
% \newcommand{\hhoo}{H$_2$O$_2$}
% \newcommand{\hhoor}{H$_2$-O$_2$}
%
\newcommand{\degs}{^\circ}
% \newcommand{\Jpkmol}{\frac{J}{kmol}}
% \newcommand{\JpkmolK}{\frac{J}{kmol K}}
% \newcommand{\Jpkg}{\frac{J}{kg}}
% \newcommand{\JpkgK}{\frac{J}{kg K}}

\newcommand{\ra}{\rightarrow}
\newcommand{\Ra}{\Rightarrow}
\newcommand{\LRa}{\Longrightarrow}
\newcommand{\lra}{\longrightarrow}

\newcommand{\pe}{\,{\scriptstyle +}\!\!=}
\newcommand{\me}{\,{\scriptstyle -}\!\!=}
\newcommand{\Var}{\textrm{Var}}
\newcommand{\Cov}{\textrm{Cov}}
\newcommand{\diag}{\textrm{diag}}

\newcommand{\etal}{\textit{et al.}}

% \newcommand{\dkl}

\def\sgn{\mathop{\rm sgn}}
\newcommand{\argmax}{\operatornamewithlimits{argmax}}
\newcommand{\argmin}{\operatornamewithlimits{argmin}}
\newcommand{\DKL}{D_{\mathrm{KL}}}

\newcommand{\iid}{\stackrel{\textrm{iid}}{\sim}}
\newcommand{\ti}[1]{\textbf{Title: }\textit{{#1}}}

\newcommand{\alf}{Alfv\'{e}n}
\newcommand{\Rs}{R$_{\odot}$}


\usepackage{algorithmic}
% \usepackage{algpseudocode}



\linenumbers

\title{Scalable Stein Variational Gradient Descent via Adaptive Spectral Delta Kernel Learning and Multifidelity Likelihoods}
\author{}
\date{\today}

\begin{document}

\maketitle
\section*{Abstract}

\section{Introduction}

\section{Related Work}

\begin{enumerate}
    \item Projected SVGD \cite{chen_projected_2020}

    \item Spectral Delta Kernels \cite{lazaro-gredilla_sparse_2010}

    \item Active Subspace based reduction for BNNs \cite{jantre_learning_2023}

    \item Continuous Normalizing Flows \cite{grathwohl_ffjord_2018}

    \item Kernelized Normalizing Flows \cite{english_kernelised_2024}

    \item SVGD as Gradient Flow \cite{liu_stein_2017}

    \item Stein breakdown in high dimensions \cite{ba_towards_2019}

    \item Stein's Method for High Dimensions \cite{chang_kernel_2020} - while the paper is almost unreadable, the authors try to use ideas from score matching such as Anneal-SGLD on the vanilla SVGD procedure and find that the kernel choice is inadequate because the bandwidth is not changing with respect to the noise level. Their proposed kernel includes an autoencoder for dimensionality reduction and conditions the hyperparameters on $\sigma$.

    \item Multilevel SVGD \cite{alsup_multilevel_2022}


    \item Alternate forms of the Stein Operator based on divergence \textcolor{red}{ours!}
\end{enumerate}

\section{Methodology}

\subsection{Bifidelity Model / Surrogate SVGD}

Consider high-fidelity and low-fidelity models with their corresponding likelihood:

\begin{align}
 L_0 &= p(y_0 | x, \theta) = \prod_{i=1}^{N} p(y_0^i | x, \theta) \\
 L_1 &= p(y_1 | x, \theta) = \prod_{i=1}^{N} p(y_1^i | x, \theta)
\end{align}


Then assuming an additive correction, a simple Stein update for the parameters $\theta_i$ would become:

\begin{equation}
    \theta_i \leftarrow \theta_i + \epsilon \phi^{\ast}(\theta)
\end{equation}

where 

\begin{equation}
    \phi^{\ast}(\theta) = \frac{1}{M}\sum_{j=1}^{M}[k(\theta_j, \theta) \grad_{\theta_j}[\log (L_{1}(\theta_j) + L_{\Delta}(\theta_j) + \log p(\theta)] + \grad_{\theta_j} k(\theta_j, \theta)]
\end{equation}

But this is not very satisfying, for one it requires us to come up with a model for $L_{\Delta}$, as is the need for pilot samples to correlate the two models.

\subsection{Generalized Multifidelity SVGD}

We treat the Stein updates as scalar QoIs. Then high-fidelity particle positions i.e. those generated using the likelihood of the high-fidelity model are given by:

\begin{equation*}
    \theta^{(0)}_i \leftarrow \theta^{(0)}_i + \epsilon \phi^{\ast}(\theta^{(0)})
\end{equation*}

And similarly, for lower-fidelity models indexed from $1$ to $m$,

\begin{equation*}
    \theta^{(j)}_i \leftarrow \theta^{(j)}_i + \epsilon \phi^{\ast}(\theta^{(j)}), \quad j=1, \cdots, m
\end{equation*}

\subsection{Particle assignment to various fidelities}

\emph{Method 1: Randomized Assignment}


\noindent \emph{Method 2: Resource allocation using model correlations}

\subsection{Graph Network Based Propagation}
Deep learning methods that model arbitrary relational structures between elements, such as graph networks are surveyed in \cite{battaglia_relational_2018}. 

Graph networks (GNs) compute interactions between all or a subset of entities and update the dynamics over multiple such blocks which accept a graph as input and return a graph with updated node features as the output. 
A key advantage offered by graph-based methods apart from the inherent invariance to permutations is the ability to operate on graphs of different sizes and shapes. 
This feature translates to so-called `discretization invariance' in expressive methods for learning PDEs such as neural operators \citep{Li2020} which exhibit good generalization properties even when trained on low-resolution data. GNs have been successfully used to simulate complex physics with long rollout times, for e.g. \cite{sanchez-gonzalez_learning_2020}.

A key drawback, of course, is that any such deep learning architecture is not enforcing the dynamics to step the particles in the direction of the KL-divergence. 
It is also unclear if the discretization-invariance property grants any advantages in representing the posterior distribution either explicitly or through varying numbers of particles. One possibility is exploiting methods such as \cite{feng_learning_2017} to train a neural simulator with better inductive bias such as that offered by GNs.


\section{Results}

\section{Conclusions and Extensions}



\bibliography{local,references}


\section*{Appendix}

\end{document}

