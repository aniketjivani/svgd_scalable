% SIAM Article Template
\documentclass[review,onefignum,onetabnum]{siamonline171218}

% Information that is shared between the article and the supplement
% (title and author information, macros, packages, etc.) goes into
% ex_shared.tex. If there is no supplement, this file can be included
% directly.

% SIAM Shared Information Template
% This is information that is shared between the main document and any
% supplement. If no supplement is required, then this information can
% be included directly in the main document.


% Packages and macros go here
\usepackage{lipsum}
\usepackage{amsfonts}
\usepackage{graphicx}
\usepackage{epstopdf}
\usepackage{soul}
\usepackage{caption}
\usepackage{subcaption}
\usepackage{algorithm}
\usepackage{algpseudocode}
\usepackage{enumitem}
\usepackage{comment}
\ifpdf
  \DeclareGraphicsExtensions{.eps,.pdf,.png,.jpg}
\else
  \DeclareGraphicsExtensions{.eps}
\fi

% \usepackage[dvipsnames,x11names]{xcolor}

\usepackage{booktabs}
\usepackage{multirow}
\usepackage{annotate-equations}

% Prevent itemized lists from running into the left margin inside theorems and proofs
\usepackage{enumitem}
\setlist[enumerate]{leftmargin=.5in}
\setlist[itemize]{leftmargin=.5in}

% Add a serial/Oxford comma by default.
\newcommand{\creflastconjunction}{, and~}

% Used for creating new theorem and remark environments
\newsiamremark{remark}{Remark}
\newsiamremark{hypothesis}{Hypothesis}
\crefname{hypothesis}{Hypothesis}{Hypotheses}
\newsiamthm{claim}{Claim}

% Sets running headers as well as PDF title and authors
% \headers{Utilizing Covariance Uncertainty in Multi-fidelity Estimation with Approximate Control Variates}{T. Coons, A. Jivani, and X. Huan}

\headers{Multifidelity SVGD}{A. Jivani, T. Coons and X. Huan}

% Title. If the supplement option is on, then "Supplementary Material"
% is automatically inserted before the title.
% \title{Utilizing Covariance Uncertainty in Multi-fidelity Estimation with Approximate Control Variates\thanks{Submitted to the editors DATE.
% \funding{Enter funding sources!}}}
\title{Scalable Stein Variational Gradient Descent via Multifidelity Particle Transport with Application to Bayesian Inference\thanks{Submitted to the editors DATE.
\funding{This work is supported in part by the National Science Foundation Graduate Research Fellowship under Grant No. DGE 1841052, the Department of Navy award N00014-23-1-2735 issued by the Office of Naval Research, and through computational resources and services provided by Advanced Research Computing at the University of Michigan, Ann Arbor.}}}

% Authors: full names plus addresses.
\author{Aniket Jivani\thanks{Department of Mechanical Engineering, University of Michigan, Ann Arbor, MI 48109 
  (\email{ajivani@umich.edu}, \url{https://me.engin.umich.edu/}).}
\and Thomas E. Coons\footnotemark[2]
\and Xun Huan\footnotemark[2]}

\usepackage{amsopn}
\DeclareMathOperator{\diag}{diag}


%%%% HELPER CODE FOR DEALING WITH EXTERNAL REFERENCES ON OVERLEAF
% (from an answer by cyberSingularity at http://tex.stackexchange.com/a/69832/226)
%%%
\makeatletter
\newcommand*{\addFileDependency}[1]{% argument=file name and extension
  \typeout{(#1)}% latexmk will find this if $recorder=0 (however, in that case, it will ignore #1 if it is a .aux or .pdf file etc and it exists! if it doesn't exist, it will appear in the list of dependents regardless)
  \@addtofilelist{#1}% if you want it to appear in \listfiles, not really necessary and latexmk doesn't use this
  \IfFileExists{#1}{}{\typeout{No file #1.}}% latexmk will find this message if #1 doesn't exist (yet)
}
\makeatother

\newcommand*{\myexternaldocument}[1]{%
    \externaldocument{#1}%
    \addFileDependency{#1.tex}%
    \addFileDependency{#1.aux}%
}
%%% END HELPER CODE

%%% Local Variables: 
%%% mode:latex
%%% TeX-master: "ex_article"
%%% End: 



% % SIAM Shared Information Template
% % This is information that is shared between the main document and any
% % supplement. If no supplement is required, then this information can
% % be included directly in the main document.


% % Packages and macros go here
% \usepackage{lipsum}
% \usepackage{amsfonts}
% \usepackage{graphicx}
% \usepackage{epstopdf}
% \usepackage{algorithmic}
% \ifpdf
%   \DeclareGraphicsExtensions{.eps,.pdf,.png,.jpg}
% \else
%   \DeclareGraphicsExtensions{.eps}
% \fi

% % Prevent itemized lists from running into the left margin inside theorems and proofs
% \usepackage{enumitem}
% \setlist[enumerate]{leftmargin=.5in}
% \setlist[itemize]{leftmargin=.5in}

% % Add a serial/Oxford comma by default.
% \newcommand{\creflastconjunction}{, and~}

% % Used for creating new theorem and remark environments
% \newsiamremark{remark}{Remark}
% \newsiamremark{hypothesis}{Hypothesis}
% \crefname{hypothesis}{Hypothesis}{Hypotheses}
% \newsiamthm{claim}{Claim}

% % Sets running headers as well as PDF title and authors
% \headers{Multi-fidelity SVGD}{A. Jivani, T. Coons, X. Huan}

% % Title. If the supplement option is on, then "Supplementary Material"
% % is automatically inserted before the title.
% % \title{An Example Article\thanks{Submitted to the editors DATE.
% % \funding{This work was funded by the Fog Research Institute under contract no.~FRI-454.}}}

% \title{Scalable Stein Variational Gradient Descent via Multifidelity Particle Transport with Application to Bayesian Inference\thanks{Submitted to the editors DATE.
% \funding{This work is supported in part by the National Science Foundation Graduate Research Fellowship under Grant No. DGE 1841052, the Department of Navy award N00014-23-1-2735 issued by the Office of Naval Research, and through computational resources and services provided by Advanced Research Computing at the University of Michigan, Ann Arbor.}}}

% % Authors: full names plus addresses.
% \author{Aniket Jivani \thanks{Department of Mechanical Engineering, University of Michigan, Ann Arbor, MI, 48109 
%   (Correspondence: \email{ajivani@umich.edu}).}
% \and Thomas E. Coons \footnotemark[2]
% \and Xun Huan \footnotemark[2]}

% \usepackage{amsopn}
% \DeclareMathOperator{\diag}{diag}


% %%%% HELPER CODE FOR DEALING WITH EXTERNAL REFERENCES ON OVERLEAF
% % (from an answer by cyberSingularity at http://tex.stackexchange.com/a/69832/226)
% %%%
% \makeatletter
% \newcommand*{\addFileDependency}[1]{% argument=file name and extension
%   \typeout{(#1)}% latexmk will find this if $recorder=0 (however, in that case, it will ignore #1 if it is a .aux or .pdf file etc and it exists! if it doesn't exist, it will appear in the list of dependents regardless)
%   \@addtofilelist{#1}% if you want it to appear in \listfiles, not really necessary and latexmk doesn't use this
%   \IfFileExists{#1}{}{\typeout{No file #1.}}% latexmk will find this message if #1 doesn't exist (yet)
% }
% \makeatother

% \newcommand*{\myexternaldocument}[1]{%
%     \externaldocument{#1}%
%     \addFileDependency{#1.tex}%
%     \addFileDependency{#1.aux}%
% }
% %%% END HELPER CODE

% %%% Local Variables: 
% %%% mode:latex
% %%% TeX-master: "ex_article"
% %%% End: 


% Optional PDF information
\ifpdf
\hypersetup{
  pdftitle={An Example Article},
  pdfauthor={D. Doe, P. T. Frank, and J. E. Smith}
}
\fi

% The next statement enables references to information in the
% supplement. See the xr-hyperref package for details.

%% Use \myexternaldocument on Overleaf
% \myexternaldocument{ex_supplement}

% FundRef data to be entered by SIAM
%<funding-group>
%<award-group>
%<funding-source>
%<named-content content-type="funder-name"> 
%</named-content> 
%<named-content content-type="funder-identifier"> 
%</named-content>
%</funding-source>
%<award-id> </award-id>
%</award-group>
%</funding-group>

% \documentclass[12pt]{article}
% \usepackage{graphicx} % Required for inserting images
% \usepackage[utf8]{inputenc}

\usepackage{hyperref}
\usepackage{xcolor}

\hypersetup{
  colorlinks=true,
  citecolor=blue,
  linkcolor=red,
  urlcolor=magenta,
  }
  % citebordercolor=false}

\usepackage{float}
\usepackage{url}
\usepackage{tabularx}
% \usepackage{amsmath}
% \usepackage{amsfonts}
% \usepackage{amssymb}
\usepackage{mathtools}
\usepackage{xurl}
\usepackage{lineno}
% \usepackage{amsthm}
\usepackage{epsfig}
% \usepackage[linesnumbered, ruled]{algorithm2e}

\usepackage{ulem}
\newcommand{\stkout}[1]{\ifmmode\text{\sout{\ensuremath{#1}}}\else\sout{#1}\fi}
\usepackage{multirow}
\usepackage{longtable}
\setlength\LTleft{0pt}
\usepackage{tabularx}
% \setlength\LTleft{0pt}
% \bibliographystyle{unsrt}
% \bibliographystyle{agu}
% \bibliographystyle{plainnat}
\usepackage{epsfig}
% \usepackage[linesnumbered, ruled]{algorithm2e}
\usepackage{natbib}
% \usepackage{color}
%
% \clearpage{}%

\newcommand\prob{\mathbb{P}}

\newcommand{\xh}[1]{\textcolor{orange}{\textbf{(xh:)} #1}}
\newcommand{\tc}[1]{\textcolor{red}{\textbf{(tc:)} #1}}
\newcommand{\aj}[1]{\textcolor{magenta}{\textbf{(aj:)} #1}}

%
\newcommand{\resp}[1]{\textcolor{red}{\textbf{Response: } #1}}
\newcommand{\respc}[1]{\textcolor{red}{#1}}

\newcommand\Myperm[2][^n]{\prescript{#1\mkern-2.5mu}{}P_{#2}}
\newcommand\Mycomb[2][^n]{\prescript{#1\mkern-0.5mu}{}C_{#2}}
%
\renewcommand{\[}{\left[}
\renewcommand{\]}{\right]}
\renewcommand{\(}{\left(}
\renewcommand{\)}{\right)}

%
\newcommand{\dd}[2]{\frac{d #1}{d #2}}
\newcommand{\ddt}[1]{\frac{d #1}{d t}}
\newcommand{\ddd}[2]{\frac{d^2 #1}{d #2^2}}
\newcommand{\dddt}[1]{\frac{d^2 #1}{d t^2}}
\newcommand{\pp}[2]{\frac{\partial #1}{\partial #2}}
\newcommand{\ppp}[2]{\frac{\partial^2 #1}{\partial #2^2}}
\newcommand{\pppp}[2]{\frac{\partial^3 #1}{\partial #2^3}}
\newcommand{\ppppp}[2]{\frac{\partial^4 #1}{\partial #2^4}}

\newcommand{\adj}[1]{#1^{*}}
\newcommand{\abs}[1]{\left|#1\right|}
\newcommand{\divergence}[1]{\nabla \cdot #1}
\newcommand{\enorm}[1]{\vvvert #1 \vvvert}
\newcommand{\grad}[1]{\nabla #1}
\newcommand{\laplace}[1]{\nabla^2 #1}
\newcommand{\norm}[2]{\left\|\, #1 \,\right\|_{#2}}
\newcommand{\order}[1]{\mathcal{O}\(#1\)}
\newcommand{\supp}{\mathop{\mathrm{supp}}}
\newcommand{\vvvert}{|\kern-1pt|\kern-1pt|}

\newcommand{\eq}[1]{\mathop{\,{\buildrel #1 \over =}\,}}
\newcommand{\ap}[1]{\mathop{\,{\buildrel #1 \over \approx}\,}}

%

%
\newcommand{\hg}{\hat{g}}
\newcommand{\hh}{\hat{h}}
\newcommand{\hi}{\hat{i}}
\newcommand{\hj}{\hat{j}}
\newcommand{\hk}{\hat{k}}
\newcommand{\hm}{\hat{m}}
\newcommand{\hn}{\hat{n}}
\newcommand{\hs}{\hat{s}}
\newcommand{\hu}{\hat{u}}
\newcommand{\hv}{\hat{v}}
\newcommand{\hx}{\hat{x}}

\newcommand{\hA}{\hat{A}}
\newcommand{\hC}{\hat{C}}
\newcommand{\hI}{\hat{I}}
\newcommand{\hJ}{\hat{J}}
\newcommand{\hN}{\hat{N}}
\newcommand{\hT}{\hat{T}}
\newcommand{\hU}{\hat{U}}

\newcommand{\hbf}{\boldsymbol{\hat{f}}}
\newcommand{\hbg}{\boldsymbol{\hat{g}}}
\newcommand{\hsig}{\hat{\sigma}}

%
\newcommand{\barg}{\bar{g}}
\newcommand{\barh}{\bar{h}}
\newcommand{\barm}{\bar{m}}
\newcommand{\barv}{\bar{v}}
\newcommand{\barw}{\bar{w}}
\newcommand{\barx}{\bar{x}}
\newcommand{\bary}{\bar{y}}

\newcommand{\barB}{\bar{B}}
\newcommand{\barC}{\bar{C}}
\newcommand{\barH}{\bar{H}}
\newcommand{\barJ}{\bar{J}}
\newcommand{\barN}{\bar{N}}
\newcommand{\barR}{\bar{R}}

\newcommand{\barmu}{\bar{\mu}}

%
\newcommand{\tf}{\tilde{f}}
\newcommand{\thh}{\tilde{h}}

\newcommand{\tA}{\tilde{A}}
\newcommand{\tg}{\tilde{g}}
\newcommand{\tH}{\tilde{H}}
\newcommand{\tJ}{\tilde{J}}
\newcommand{\tQ}{\tilde{Q}}
\newcommand{\tT}{\tilde{T}}

\newcommand{\tmu}{\tilde{\mu}}
\newcommand{\ttheta}{\tilde{\theta}}
\newcommand{\txi}{\tilde{\xi}}

\newcommand{\tTheta}{\tilde{\Theta}}
\newcommand{\tXi}{\tilde{\Xi}}

%
\newcommand{\mb}[1]{\mathbf{#1}}
\newcommand{\sbf}[1]{\boldsymbol{#1}}

\newcommand{\bb}{\textbf{b}}
\newcommand{\bd}{\textbf{d}}
\newcommand{\bee}{\textbf{e}}
\newcommand{\bff}{\textbf{f}}
\newcommand{\bh}{\textbf{h}}
\newcommand{\bg}{\textbf{g}}
\newcommand{\bk}{\textbf{k}}
\newcommand{\bii}{\textbf{i}}
\newcommand{\bj}{\textbf{j}}
\newcommand{\bl}{\textbf{l}}
\newcommand{\bn}{\textbf{n}}
\newcommand{\bp}{\textbf{p}}
\newcommand{\br}{\textbf{r}}
\newcommand{\bs}{\textbf{s}}
\newcommand{\bt}{\textbf{t}}
\newcommand{\bu}{\textbf{u}}
\newcommand{\bv}{\textbf{v}}
\newcommand{\bw}{\textbf{w}}
\newcommand{\bx}{\textbf{x}}
\newcommand{\by}{\textbf{y}}

\newcommand{\bA}{\mathbf{A}}
\newcommand{\bC}{\mathbf{C}}
\newcommand{\bE}{\mathbf{E}}
\newcommand{\bF}{\mathbf{F}}
\newcommand{\bG}{\mathbf{G}}
\newcommand{\bI}{\mathbf{I}}
\newcommand{\bK}{\mathbf{K}}
\newcommand{\bN}{\mathbf{N}}
\newcommand{\bQ}{\mathbf{Q}}
\newcommand{\bR}{\mathbf{R}}
\newcommand{\bT}{\mathbf{T}}
\newcommand{\bU}{\mathbf{U}}
\newcommand{\bV}{\mathbf{V}}
\newcommand{\bY}{\mathbf{Y}}

\newcommand{\balpha}{\boldsymbol{\alpha}}
\newcommand{\bbeta}{\boldsymbol{\beta}}
\newcommand{\bepsilon}{\boldsymbol{\epsilon}}
\newcommand{\bhsig}{\boldsymbol{\hsig}}
\newcommand{\bpsi}{\boldsymbol{\psi}}
\newcommand{\bsig}{\boldsymbol{\sigma}}
\newcommand{\btau}{\boldsymbol{\tau}}
\newcommand{\bmu}{\boldsymbol{\mu}}
\newcommand{\btheta}{\boldsymbol{\theta}}
\newcommand{\bphi}{\boldsymbol{\phi}}
\newcommand{\bxi}{\boldsymbol{\xi}}

\newcommand{\bDelta}{\boldsymbol{\Delta}}
\newcommand{\bTheta}{\boldsymbol{\Theta}}
\newcommand{\bXi}{\boldsymbol{\Xi}}
\newcommand{\bOmega}{\boldsymbol{\Omega}}
\newcommand{\bSigma}{\boldsymbol{\Sigma}}

%
\newcommand{\EE}{\mathbb{E}}
\newcommand{\II}{\mathbb{I}}
\newcommand{\NN}{\mathbb{N}}
\newcommand{\QQ}{\mathbb{Q}}
\newcommand{\PP}{\mathbb{P}}
\newcommand{\RR}{\mathbb{R}}

%
\newcommand{\vt}{\vec{t}}
\newcommand{\vu}{\vec{u}}
\newcommand{\vv}{\vec{v}}
\newcommand{\vx}{\vec{x}}

\newcommand{\vV}{\vec{V}}

\newcommand{\vo}{\vec{\omega}}

%
\newcommand{\dotk}{\dot{k}}
\newcommand{\dotm}{\dot{m}}
\newcommand{\dotx}{\dot{x}}

\newcommand{\dotomega}{\dot{\omega}}

%
\newcommand{\CA}{\mathcal{A}}
\newcommand{\CB}{\mathcal{B}}
\newcommand{\CD}{\mathcal{D}}
\newcommand{\CE}{\mathcal{E}}
\newcommand{\CF}{\mathcal{F}}
\newcommand{\CG}{\mathcal{G}}
\newcommand{\CH}{\mathcal{H}}
\newcommand{\CJ}{\mathcal{J}}
\newcommand{\CK}{\mathcal{K}}
\newcommand{\CL}{\mathcal{L}}
\newcommand{\CM}{\mathcal{M}}
\newcommand{\CN}{\mathcal{N}}
\newcommand{\CQ}{\mathcal{N}}
\newcommand{\CR}{\mathcal{R}}
\newcommand{\CS}{\mathcal{S}}
\newcommand{\CT}{\mathcal{T}}
\newcommand{\CU}{\mathcal{U}}
\newcommand{\CV}{\mathcal{V}}
\newcommand{\CW}{\mathcal{W}}
\newcommand{\CX}{\mathcal{X}}
\newcommand{\CY}{\mathcal{Y}}

\newcommand{\CbarJ}{\bar{\mathcal{J}}}
\newcommand{\CbarL}{\bar{\mathcal{L}}}
\newcommand{\CbarR}{\bar{\mathcal{R}}}

%
\newcommand{\du}{\delta{u}}

\newcommand{\dbeta}{\delta{\beta}}
\newcommand{\dxi}{\delta{\xi}}
\newcommand{\deta}{\delta{\eta}}
\newcommand{\drho}{\delta{\rho}}
\newcommand{\dtau}{\delta{\tau}}

\newcommand{\dbu}{\delta{\boldsymbol{u}}}
\newcommand{\dbp}{\delta{\boldsymbol{p}}}
\newcommand{\dbx}{\delta{\boldsymbol{x}}}

\newcommand{\Dx}{\Delta{x}}
\newcommand{\Dy}{\Delta{y}}
\newcommand{\Dt}{\Delta{t}}

%
\newcommand{\myblue}[1]{{\color[rgb]{0,0,0.65} #1}}
\newcommand{\mygreen}[1]{{\color[rgb]{0,.65,0} #1}}
\newcommand{\mywhite}[1]{{\color[rgb]{1.0,1.0,1.0} #1}}
\newcommand{\myred}[1]{{\color[rgb]{0.65,0.0,0.0} #1}}
\newcommand{\myblack}[1]{{\color[rgb]{0.0,0.0,0.0} #1}}
\newcommand{\mygrey}[1]{{\color[rgb]{0.6,0.6,0.6} #1}}

%
% \newcommand{\coo}{CO$_2$}
% \newcommand{\hho}{H$_2$O}
% \newcommand{\oo}{O$_2$}
% \newcommand{\nn}{N$_2$}
% \newcommand{\mwe}{MW$_e$}
% \newcommand{\nox}{NO$_{\textrm{x}}$}
% \newcommand{\cooe}{CO$_{2e}$}
% \newcommand{\nno}{N$_2$O}
% \newcommand{\noo}{NO$_2$ }
% \newcommand{\chhhh}{CH$_4$ }
% \newcommand{\hhoo}{H$_2$O$_2$}
% \newcommand{\hhoor}{H$_2$-O$_2$}
%
\newcommand{\degs}{^\circ}
% \newcommand{\Jpkmol}{\frac{J}{kmol}}
% \newcommand{\JpkmolK}{\frac{J}{kmol K}}
% \newcommand{\Jpkg}{\frac{J}{kg}}
% \newcommand{\JpkgK}{\frac{J}{kg K}}

\newcommand{\ra}{\rightarrow}
\newcommand{\Ra}{\Rightarrow}
\newcommand{\LRa}{\Longrightarrow}
\newcommand{\lra}{\longrightarrow}

\newcommand{\pe}{\,{\scriptstyle +}\!\!=}
\newcommand{\me}{\,{\scriptstyle -}\!\!=}
\newcommand{\Var}{\textrm{Var}}
\newcommand{\Cov}{\textrm{Cov}}
% \newcommand{\diag}{\textrm{diag}}

\newcommand{\etal}{\textit{et al.}}

% \newcommand{\dkl}

\def\sgn{\mathop{\rm sgn}}
\def\<#1>{\mathinner{\langle#1\rangle}}
% \DeclareMathOperator*{\argmin}{arg min}
\newcommand{\argmax}{\operatornamewithlimits{argmax}}
\newcommand{\argmin}{\operatornamewithlimits{argmin}}
\newcommand{\DKL}{D_{\mathrm{KL}}}

\newcommand{\iid}{\stackrel{\textrm{iid}}{\sim}}
\newcommand{\ti}[1]{\textbf{Title: }\textit{{#1}}}

\newcommand{\alf}{Alfv\'{e}n}
\newcommand{\Rs}{R$_{\odot}$}


% \usepackage{algorithmic}
% \usepackage{algpseudocode}
\linenumbers

% \title{Scalable Stein Variational Gradient Descent via Nested Multifidelity Transport}
% \author{Aniket Jivani, Thomas Coons, Xun Huan}
% \date{\today}

\begin{document}
\maketitle

\begin{abstract}
Stein's method was originally developed to approximate moment computations for special distribution families in statistics using a particular class of linear operators. 
Various advancements in algorithms leveraging Stein operators have since found application in problems in hypothesis testing, information theory, optimal transport and inference. 
A simple method for performing Bayesian Inference, called Stein Variational Gradient Descent (SVGD), relies on gradient-based updates using kernelized Stein discrepancy to deterministically transport particles and approximate a posterior distribution of model parameters in the process. 
This is essentially non-parametric variational inference that is capable of approximating complex posterior distributions. However, SVGD struggles to accomodate many-query evaluations of the likelihood function when a computationally expensive high-fidelity model is involved. 
In addition, the quadratic complexity of the kernel function evaluations makes posterior characterization with a very large number of particles infeasible. 
In this work, we introduce a multifidelity version of SVGD that brings down the overall cost of particle position updates by leveraging computationally inexpensive lower-fidelity models. 
These can take the form of forward models with tractable likelihoods, or tractable density estimation procedures on the high-fidelity model. 
The key contribution is a multifidelity gradient estimator that partitions the incremental transform to enable evaluations under different likelihood models and also optionally partitions the kernel evaluations based on the neighbourhood of influence for each particle. These advancements greatly accelerate Bayesian inference via SVGD and can also be used to propose an estimator for expected information gain that implicitly incorporates multi-fidelity information.
\end{abstract}
% We also propose a sparsified kernel evaluation method that borrows ideas from sparse attention techniques for transformers to bring down the quadratic complexity of computing pairwise distances.


\begin{keywords}
  example, \LaTeX
\end{keywords}

% REQUIRED
\begin{AMS}
    62K99, 68T07
\end{AMS}


\section{Introduction}

\section{Background and Related Work}


% \begin{enumerate}
%     \item Spectral Delta Kernels \cite{lazaro-gredilla_sparse_2010}

%     \item Active Subspace based reduction for BNNs \cite{jantre_learning_2023}

%     \item Continuous Normalizing Flows \cite{grathwohl_ffjord_2018}

%     \item Kernelized Normalizing Flows \cite{english_kernelised_2024}

%     \item SVGD as Gradient Flow \cite{liu_stein_2017}

%     \item Stein breakdown in high dimensions \cite{ba_towards_2019}

%     % \item Stein's Method for High Dimensions \cite{chang_kernel_2020} - while the paper is almost unreadable, the authors try to use ideas from score matching such as Anneal-SGLD on the vanilla SVGD procedure and find that the kernel choice is inadequate because the bandwidth is not changing with respect to the noise level. Their proposed kernel includes an autoencoder for dimensionality reduction and conditions the hyperparameters on $\sigma$.

%     \item Multilevel SVGD \cite{alsup_multilevel_2022}

%     \item Sparse Sinkhorn Attention \cite{tay_sparse_2020}
% \end{enumerate}
\subsection{Stein Variational Gradient Descent}\label{ss: svgd}

Bayesian inference is a rigorous and general framework for reasoning under uncertainty: we construct a full probability model for all observable and unobservable quantities in a problem, followed by conditioning on the observed data to characterize the posterior distribution for the unobserved quantities. We provide a brief description below:

Let $\theta$ denote an unobservable quantity or parameter, and $D$ denote data or \textit{observations} collected through experiments or simulations. In order to estimate $\theta$, we define and decompose a joint probability distribution on $\theta$ and $D$ as follows:
\begin{align}
    p(\theta, D) = \underbrace{p(\theta)}_{\text{prior}} \underbrace{p(D\mid\theta)}_{\text{likelihood}}
\end{align}

The object of interest in Bayesian inference is the ``posterior density'' $p(\theta \mid D)$, which is the updated uncertainty for the unknown parameters conditioned on the newly acquired observations. It is obtained through the re-arrangement of the various conditional probabilities and summarized via Bayes' theorem:

\begin{align}
    p(\theta \mid D) = \frac{p(D\mid\theta) \, p(\theta)}{p(D)},
    \label{eqn: bayes_rule_v1}
\end{align}

where $p(D)$ is the evidence or marginal likelihood, and often intractable quantity that acts as a normalizing constant on the often tractable likelihood and prior.

As far as inference techniques go, Markov Chain Monte Carlo (MCMC) methods are considered a workhorse of modern statistical inference. 
The high-level idea is to generate a sequence of random samples through construction of a Markov chain whose equilibrium distribution is the target posterior distribution. 
In addition to sampling from arbitrary distribution families, it also has the key benefit of asymptotic convergence guarantees. 
However, these are computationally intensive when dealing with high-dimensional parameter spaces, despite some impressive algorithmic advancements that make use of gradient information to guide new proposals.

Variational Inference (VI) reformulates traditional Bayesian inference as an optimization problem, where the posterior is approximated by a parametric family of distributions that is closest to the true posterior through a suitable measure. This is usually the reverse KL divergence or other measures from the family of $f$-divergence between probabiliy distributions. 
While VI is much more scalable to higher-dimensional spaces compared to MCMC, this scalability often trades off its approximation capacity, which is constrained by the choice of the variational family. 
For instance, mean-field variational Bayes assumes factorization of the variational distribution into independent variational approximations over each latent variable. 
However, dependencies between the hidden variables mean that this family typically does not contain the true posterior distribution. 
In general, VI can result in biased approximations to the posterior.


Stein Variational Gradient Descent (SVGD), introduced in \cite{Liu2016}, proposes a powerful inference method. First, a transport map $T$ is defined between a tractable reference distribution $q_0(x)$ and the target density. 
Rather than restricting the set of transforms $T$ to a certain parametric form, it is instead constructed incrementally to propagate an initial set of particles drawn from $q_0$, with each successive transform directed towards steepest rate of change of KL divergence. 
This is one example of a class of methods that make use of gradient flows of various divergence measures to sample from a target density in generative modeling. SVGD in particular considers gradient flow of KL divergence over the $\CH$-Wasserstein distance, where $\CH$ is an RKHS equipped with kernel $k(x, x')$. This is exploited to result in a deterministic, interpretable and simple inference procedure resembling gradient descent. Notably, the convergence to the target density / fixed point of the system is decoupled from the update procedure, and various strategies can be used to accelerate the optimization, for instance, preconditioning gradient descent with Hessian information \citep{wang_stein_2019}. 

While the procedure is naturally sensitive to choices of initial conditions and optimization methods used to find the fixed point, it can be used to characterize complex posterior distributions and is easily parallelizable. 

Vanilla SVGD takes as input the target density function $p(x)$ and a set of initial particles $\{x_i^0\}_{i=1}^{n} \sim q_0(x)$ and returns a set of particles $\{x_i^L\}_{i=1}^{n}$ after L iterations that approximates the target distribution.

For iteration $l$, we update the particle positions via:

\begin{align}
    x_i^{l + 1} &\leftarrow x_i^l + \epsilon_l\hat{\phi}^{\ast}(x_i^l)\nonumber\\
    \text{ where }\hat{\phi}^{\ast}(x) &= \frac{1}{n}\left[\sum_{j=1}^n k(x_j^l, x) \nabla_{x_j^l}\log p(x_j^l) + \nabla_{x_j^l}k(x_j^l, x)\right]
    \label{eq: vanilla_svgd}
\end{align}

The first term drives the particles towards regions towards the high probability regions of the posterior distribution, while the second term acts as a repulsive force that prevents mode collapse.

Generally, there are two major costs associated with performing SVGD, namely:

\begin{enumerate}
    \item Computing the gradient $\nabla_x \log p(x)$. For Bayesian Inference, this amounts to evaluating the score function associated with the posterior distribution and thereby the cost of repeated likelihood evaluations via a high-fidelity model.

    \item The $O(n^2)$ cost associated with computing elements $k(x_i, x_j)$ of the dense kernel matrix.
\end{enumerate}

The kernel weighting factor as well as the particle repulsion term also have scope for cost savings through identification of covariance structures as the map is iteratively constructed. 

One simple example is propagating an initial set of particles towards a bimodal target density. 
Eventually, particles will assume memberships of separate clusters such that a particle belonging to one cluster will exhibit little to no influence on those belonging to the other cluster. This introduces some structure and potentially sparsity in the covariance matrix. 



Estimation of sparse covariance structures \citep{fop_model-based_2018} is of great interest in the statistical literature and in studies of  graphical models. For instance, a message-passing version of SVGD \citep{wang_stein_2018} is tailored to continuous graphical models through localized kernel approximations. 

However, we will initially focus on efficient evaluation of the score function first through multi-fidelity setups, before considering a potential combination of the kernel approximation and score approximation ideas.

\subsection{Approximate Control Variates for Multi-fidelity Estimation}\label{ss: acv}

\textcolor{red}{Stealing the text from acv with uq with some cuts}

Consider a mapping $Q=f_0(Z)$ that relates the random vector $Z \in \mathbb{R}^{n_{z}}$ and the random variable $Q \in \mathbb{R}$. The expected value of $Q$ is often approximated by a standard $N$-sample Monte Carlo (MC) estimator:
\begin{align}\label{eqn:mc}
    \mathbb{E}\left[Q_0\right] 
    \approx \hat{Q}_0 := \frac{1}{N} \sum^{N}_{j=1}f_0(z^{(j)}),
\end{align}

where $z^{(j)} \sim p(Z)$ are independent and identically distributed (i.i.d.) samples drawn from the probability distribution of $Z$. The MC estimator $\hat{Q}$ is unbiased, making its estimator error contributed entirely by its variance $N^{-1}\Var[Q]$. When using computationally intensive models, the number of samples $N$ that can be afforded may not be sufficiently high to limit this variance.

To reduce the estimator variance, multi-fidelity methods leverage an ensemble of low-fidelity models whose outputs are correlated with the high-fidelity model of interest but with diminished costs. 

Control variates (CVs) are a widely used classical method used as a variance reduction technique for Monte-Carlo methods. 
The general approach is to identify another function $f_1$, such that the traditional expected value estimator described in \autoref{eqn:mc} can be replaced by one constructed with $f_0 - f_1$, with smaller variance. The function $f_1$ is then referred to as a control variate. 
Typically, selection of $f_1$ and identifying a rich set of candidate CVs is a non-trivial procedure barring the exceptions where CVs can be identified from domain knowledge. 
Over the years, linear transformations of the score function have been a popular choice given the high canonical correlations with the target that result in better variance reduction, and their straightforward incorporation into modern MCMC methods that leverage gradient information \citep{papamarkou_zero_2014}. \cite{si_scalable_2021} extends CV identification to the high-dimensional settings by proposing a variational formulation based on Stein operators.


In the context of multifidelity methods however, we typically already have access to a hierarcy of models, where each available low-fidelity model can be leveraged as a control variate. The approximate control variate (ACV) technique \cite{Gorodetsky2020, bomarito_optimization_2022} provides a rigorous formulation for deriving estimators in this setting.
This method introduces $M$ auxiliary random variables $Q_{m}$ for $m=1,\ldots,M$ that are outputs of the corresponding (e.g., low-fidelity) models $Q_{m} =f_{m}(Z)$, then combines the corresponding MC estimates of each low-fidelity model, $\hat{Q}_{m}$, into a single estimator. The ACV estimator, which we denote $\tilde{Q}$, can be written as:
\begin{align}
        \Tilde{Q}(z,\alpha,\CA) &:= 
        \hat{Q}_{0}(z_{0})+\sum^{M}_{m=1}\alpha_{m}\left( \hat{Q}_{m}(z^{\ast}_{m})-\hat{\mu}_{m}(z_{m}) \right) \nonumber\\ 
        &= \hat{Q}_{0}(z_{0})+\sum^{M}_{m=1}\alpha_{m}\left( \hat{Q}_{m}(z^{\ast}_{m})-\hat{Q}_{m}(z_{m}) \right),     \label{eqn:ACV-Formula}
\end{align}
where $\hat{\mu}_{m}$ is an MC estimate for the $m$th model mean, $\alpha = \left[ \alpha_{1}, \ldots, \alpha_M \right]$ is a vector of control variate weights, and the input samples $z$ are partitioned into subsets $z_{0}$, $z_{m}$, and  $z^{\ast}_{m}$ according to a sample partitioning strategy $\CA$. The control variate weights $\alpha$ and the sample partitioning strategy $\CA$ are hyperparameters of the estimator.
%that are chosen to optimally reduce the estimator variance. 

Since the ACV estimator is unbiased with respect to the high-fidelity model, its error is equal to the estimator variance. To express this variance mathematically, we introduce a vectorized notation for the ACV estimator:
\begin{align}\label{eqn:vectorized-acv}
    \Tilde{Q}(z;\alpha,\CA) = \hat{Q}_{0} + \alpha^{\top}\Delta,
\end{align}
where $\Delta:=\left[ \Delta_{1}(z_{1}^{\ast},z_{1}), \ldots,  \Delta_{M}(z_{M}^{\ast},z_{M})\right]$, $\Delta_{m}(z_{m}^{\ast},z_{m}):=\hat{Q}_{m}(z^{\ast}_{m})-\hat{Q}_{m}(z_{m})$, and the explicit dependence on the inputs $z_{m}$ and $z_{m}^{\ast}$ are omitted for simplicity of notation. The ACV estimator variance is then:
\begin{align}\label{eqn:acv-variance}
    \Var[\Tilde{Q}(z;\alpha,\CA)] = \Var[\hat{Q}_{0}] - \alpha^{\top}(\Cov[\Delta, \Delta])^{-1}\alpha + 2 \alpha^{\top} \Cov[\Delta, \hat{Q}_{0}].
\end{align}
The estimator variance is dependent on the covariances between each model output, the control variate weights, and the sample partitioning that influences how each $\Delta$ term covaries with each other and with $\hat{Q}_{0}$. If the exact covariance between each model output is known, then the optimal weights for a given ACV sample allocation $\CA$ can be computed by minimizing \eqref{eqn:acv-variance}. The optimal weights are:
\begin{align}\label{eqn:alpha-star-acv}
    \alpha^{\ast}(\CA)=-\Cov[\Delta, \Delta]^{-1}\Cov[\Delta, \hat{Q}_{0}],
\end{align}
and when these weights are set to their optimal values, the ACV estimator variance is~\cite{Gorodetsky2020}:
\begin{align}\label{eqn:acv-variance-opt}
    \Var[\Tilde{Q}^{\alpha^{\ast}}](\CA) = \Var[\hat{Q}_{0}] - \Cov[\Delta, \hat{Q}_{0}]^{\top} \Cov[\Delta, \Delta]^{-1}\Cov[\Delta, \hat{Q}_{0}].
\end{align}

The sample partitioning $\CA$ determines how much computational workload is distributed to each model. The well-known multi-fidelity Monte Carlo (MFMC) \cite{peherstorfer_optimal_2016} and multi-level Monte Carlo (MLMC) \cite{giles_multilevel_2015} methods can be shown to be special cases of ACV that differ in the family of possible $\CA$ choices. Solving for the optimal sample partitioning strategy $\CA^{\ast}$ is more intensive than solving for the optimal weights $\alpha^{\ast}$ but many tools \cite{bomarito_multi_2020, jakeman_pyapprox_2023} are available to approximately solve the optimization problem:
\begin{align}
    \min_{\mathcal{A}\in\mathbb{A}} &\quad \Var[\tilde{Q}(Z;\alpha^{\ast},\mathcal{A})] \label{eqn:mxmcpy}\\
    \text{subject to} &\quad \mathcal{W}(w,\mathcal{A})\leq w_{\text{budget}},\label{eqn:mxmcpy_constraint}
\end{align}
where $\mathbb{A}$ is a wide set of predefined allowable set of sample allocations, $w_{\text{budget}}$ is the total budget constraint, and $\mathcal{W}(w,\mathcal{A})$ computes the total cost of the estimator under model costs $w := [w_0, \ldots, w_M]$ and sample allocation $\CA$.

Importantly, the solutions to \eqref{eqn:mxmcpy} and \eqref{eqn:alpha-star-acv} are only computable with knowledge of the model output covariance matrix, $\Sigma = \Cov\left[ f_0(Z),\ldots,f_M(Z)
\right]$, as well as the model costs vector $w$. 
While the relationship between $\alpha$ and the estimator variance is clear from \eqref{eqn:acv-variance}, the relationship between $\Sigma$ and the estimator variance is dependent on $\CA$ via the terms in \eqref{eqn:acv-variance}:
\begin{align}
    \Cov[\Delta, \Delta] &= G(\CA) \odot \Sigma, \label{eqn:cov-delta-delta} \\
    \Cov[\Delta, \hat{Q}_{0}] &= \text{diag}(G(\CA)) \odot \text{col}_0( \Sigma ), \label{eqn:cov-delta-Q}
\end{align}
where $G \in \RR^{M\times M}$ is matrix that depends on $\CA$ and the exact sample allocation strategy in use (see \cite{bomarito_multi_2020}), $\odot$ signifies the Hadamard (elementwise) product, and $\text{col}_{0}( \Sigma )$ is the first column of $\Sigma$, i.e., the covariances between the high-fidelity model and all the low-fidelity models.
% While the model costs may often be known ahead of time, it is unlikely that the covariance matrix $\Sigma$ would be known \textit{a priori}. Traditionally, a set of $N_{\text{pilot}}$ independent pilot samples are used to compute an approximate covariance matrix $\hat{\Sigma}$ via the unbiased sample covariance formula:
% \begin{align}\label{eqn:cov}
%     \hat{\Sigma} = \frac{1}{N_{\text{pilot}}-1}\sum_{j=1}^{N_{\text{pilot}}}\left( f^{(j)} - \overline{f} \right) \left( f^{(j)} - \overline{f} \right)^{\top},
% \end{align}
% \sloppy where 
% %$N_{\text{pilot}}$ denotes the number of pilot samples, 
% $f^{(j)}=[ f_{0}(z^{(j)}),\ldots,f_{M}(z^{(j)}) ]^{\top}$, $\overline{f}=[ \overline{f}_{0},\ldots,\overline{f}_{M} ]^{\top}$, and $\overline{f}_{m}=\frac{1}{N_{\text{pilot}}}\sum_{j=1}^{N_{\text{pilot}}}f_{m}(z^{(j)})$. Too few pilot evaluations can lead to inaccurate covariance estimates and thus suboptimal ACV hyperparameters and high ACV errors. On the other hand, too many pilot evaluations can carry too much cost and leave too little computational budget left for the actual ACV estimator evaluation. While this trade-off has been investigated empirically in existing ACV literature~\cite{Peherstorfer2016, Gorodetsky2020}, a rigorous method is needed to help balance covariance uncertainty and pilot sampling budget. 



Alternative sampling-based estimators beyond the ACV formulation above, such as the multilevel best linear unbiased (MLBLUE) estimator from \cite{schaden_multilevel_2020} 
can produce impressive variance reduction like ACV. 
In fact, the generalized linear grouped ACV estimator (GACV) \cite{gorodetsky_grouped_2024} generalizes these methods. Though exact parametrizations may differ across estimators, they share common features of estimator weights and sample allocations.

% While MLBLUE and GACV are competitive with and can outperform ACV in some cases, we nonetheless use ACV in this work since potential performance improvements may be marginal relative to the issue of optimal pilot sampling. Importantly, the issue of covariance estimation still applies to these other estimators, and the pilot sampling termination algorithm presented in this paper can be applied to these other estimators by replacing the above variance formulae with their corresponding equations from \cite{gorodetsky_grouped_2024}.

\subsection{Bayesian Optimal Experimental Design}



\subsection{Flow Matching}

Below, we briefly review the current practice in conditional flow matching algorithms and motivate their use in learning maps between samples from multiple-fidelity model likelihoods.

Generative modeling tasks consider the problem of approximating and sampling from a probability distribution. For instance, continuous normalizing flows (CNFs) proposed in \cite{grathwohl_ffjord_2018} express an invertible mapping between a fixed and tractable source distribution and the data distribution using Neural Ordinary Differential Equations (NODEs). CNFs can be trained and scaled to larger datasets better through a regression of the ODE drift, a so-called Flow Matching (FM) objective. The advances proposed in \cite{tong_improving_2024} generalize the flow matching framework proposed by \cite{lipman_flow_2023} to use transport maps between arbitrary distributions and approximate the dynamic Optimal Transport problem during sampling of conditional paths to improve the efficiency of training and inference. 

A smooth time-varying vector field $u: [0, 1] \times \RR^d \to \RR^d$ is defined by the following ordinary differential equation (ODE):
\begin{equation}
    dx = u(t, x)dt
\end{equation}

Given density $p_0$ over $\RR^d$, if $\phi_t(x)$ is the solution for the above ODE (i.e. it denotes the position of point $x$ transported along $u$ from time $0$ up to time $t$), this integration map introduces a pushforward density $p_t$ ($p_t$ is the density of points $x \sim p_0$ transported along $u$ from time 0 to time $t$) that is characterized by the continuity equation:

\begin{equation}
    \frac{\partial p}{\partial t} = -\grad.\left(p_t u_t\right)
\end{equation}

We can characterize the marginal probability path $p_t$ as a mixture of conditional probability paths:

\begin{equation}
    p_t(x) = \int p_t(x | z) q(z) dz
\end{equation}

If a path $p_t(x|z)$ is generated from $u_t(x | z)$ from initial conditions $p_0(x | z)$, then the vector field $u_t(x)$ given by:

\begin{equation}
    u_t(x) = \EE_{q(z)}\left[\frac{u_t(x|z)p_t(x|z)}{p_t(x)}\right]\label{eq: marg_cond_velocity_field}
\end{equation}

generates probability path $p_t(x)$ from initial conditions $p_0(x)$.

The key ideas behind flow matching rely on this connection between marginal vector fields and marginal probability paths to decompose the intractable marginal vector field into conditional vector fields.

The original flow matching objective considered in \cite{lipman_flow_2023} is intractable for general source density $q_0$ mapping to target $q_1$:

\begin{equation}
    \CL_{\text{FM}}(\theta)= \EE_{t \sim \mathcal{U}(0, 1), x \sim p_t(x)} ||v_\theta(t, x) - u_t(x)||^2
\end{equation}

In the special case where the marginal densities $p_t(x) = \CN(x | \mu_t, \sigma_t^2)$ are Gaussian, a possible simple (but non-unique) ODE satisfies:

\begin{equation}
    \phi_t(x_0) = \mu_t + \sigma_t \left(\frac{x_0 - \mu_0}{\sigma_0}\right) \label{eq: gaussian_ode_sol}
\end{equation}

and the unique vector field whose integration map satisfies \eqref{eq: gaussian_ode_sol} has the form:

\begin{equation}
    u_t(x) = \frac{\sigma_t'}{\sigma_t} (x - \mu_t) + \mu_t' \label{eq: vector_field_gauss_dens}
\end{equation}

where $\sigma_t'$ and $\mu_t'$ denote the time derivatives of $\sigma_t$ and $\mu_t$ respectively, and the vector field $u$ with initial conditions $\CN(\mu_0, \sigma_0^2)$ generates the Gaussian probability path $p_t(x) = \CN(x | \mu_t, \sigma_t^2)$.

The flow matching loss under assumptions of existence and exchange of different integrals, and boundedness of the solution fields can be switched out for an easier-to-regress conditional flow matching loss with the loss gradient only changing by a constant factor that is independent of $\theta$:

\begin{equation}
    \CL_{\text{CFM}}(\theta) = \EE_{t, q(z), p_t(x|z)}||v_\theta(t, x) - u_t(x|z)||^2 \label{eq: cfm_loss}
\end{equation}

For all practical purposes, we can solve the much more useful CFM objective, under the condition that we can:

\begin{enumerate}
    \item Sample from $q(z)$

    \item Sample from $p_t(x|z)$

    \item Calculate $u_t(x|z)$
\end{enumerate}

\begin{equation}
    \nabla_{\theta}\CL_{\text{FM}}(\theta) =\nabla_{\theta}\CL_{\text{CFM}}(\theta) \label{eq: cfm_fm_grad}
\end{equation}

A short proof of this is provided below from \cite{tong_improving_2024}:

\begin{align*}
    \grad_{\theta}\EE_{p_t(x)}||v_\theta(t, x) - u_t(x)||^2 &= \grad_\theta \EE_{p_t(x)}(||v_\theta(t, x)||^2 - 2\langle v_\theta(t, x) u_t(x)\rangle) \quad \textcolor{red}{\text{($u_t(x)
    $ does not depend on $\theta$})}
\end{align*}


\begin{align*}
    \grad_{\theta}\EE_{q(z),p_t(x|z)}||v_\theta(t, x) - u_t(x, z)||^2 &= \EE_{q(z), p_t(x|z)}\grad_{\theta}(||v_\theta(t, x)||^2 - 2 \langle v_\theta(t, x) , u_t(x|z)\rangle)
\end{align*}

The first term in each loss function can be reparametrized as:
\begin{equation*}
    \EE_{p_t(x)}||v_\theta(t, x)||^2 = \iint ||v_\theta(t, x)||^2 p_t(x|z)q(z)dz dx = \EE_{q(z), p_t(x|z)}||v_\theta(t, x)||^2
\end{equation*}

while we substitute \eqref{eq: marg_cond_velocity_field} in the inner product expression for FM to recover the CFM inner product and thus equalize the loss gradients:

\begin{align*}
    \EE_{p_t(x)}\langle v_\theta(t, x), u_t(x)\rangle &= \int \biggl<v_\theta(t, x), \frac{\int u_t(x|z)p_t(x|z)q(z)dz}{p_t(x)}\biggr>p_t(x)dx \\
    &=\iint \langle v_\theta(t, x), u_t(x|z)\rangle p_t(x|z)q(z)dzdx \quad \textcolor{red}{\text{(inner product distributed over integral argument)}}\\
    &=\EE_{q(z), p_t(x|z)}\langle v_\theta(t, x) u_t(x | z) \rangle
\end{align*}

\medskip
\emph{I-CFM (via Independent Coupling):}
For source point $x_0$ and target point $x_1$:

\begin{enumerate}
    \item $q(z) = q(x_0)q(x_1)$

    \item $p_t(x|z) = \CN(x | tx_1 + (1 - t)  x_0, \sigma^2)$

    \item Using \eqref{eq: vector_field_gauss_dens} for the conditional vector field with $\mu_t = tx_1 + (1 - t)x_0 $ and $\sigma_t = \sigma$, $u_t(x|z) = x_1 - x_0$
\end{enumerate}

\medskip
\emph{OT-CFM (sampling via OT Map):}

The key difference from I-CFM is that instead of $x_0, x_1$ being sampled independently from their marginal distributions, they are sampled jointly according to the optimal transport map $\pi$:

$$q(z) = \pi(x_0, x_1)$$

In cases where a static OT plan is computationally infeasible to determine exactly, a minibatch OT approximation \citep{fatras_learning_2020} shows improvements over random sampling plans in terms of model performance and training times. 
Related work such as that of \cite{finlay_how_2020} use a regularized CNF with dynamic OT objectives though these are difficult to train and scale.


\emph{SB-CFM (entropy-regularized OT map):}
Recent efforts such as \citep{de_bortoli_diffusion_2021,heng_diffusion_2024} have also focused on efficient inference in diffusion models and general Bayesian computation by reformulating them as a Schrödinger bridge (SB) problem where the forward process need not be run for large number of steps to ensure $p_N = p_{\text{prior}}$. 
The static SB problem can be seen as an entropy-regularized quadratic cost OT problem that is an attractive choice for high-dimensional OT between arbitrary data distributions. 
These can be applied to an entropic variant of OT-CFM, SB-CFM, to match probability flow of a Schrödinger bridge with a Brownian motion refererence process.

The SB problem seeks a process $\pi$ that is closest to the initial time marginal $p_{\text{ref}}$ while having initial and terminal marginal distributions specified by $q(x_0)$ and $q(x_1)$ respectively i.e.:
\begin{equation*}
\pi^{\ast} = \argmin_{\pi(x_0) = q(x_0), \pi(x_1)=q(x_1)} \DKL(\pi, p_{\text{ref}})
\end{equation*}

Then the above solution can be recovered through the marginal vector field $u_t(x)$ defined via:

\begin{enumerate}
\item $q(z) = \pi_{2\sigma^2}(x_0, x_1)$ where $\pi_{2\sigma^2}(x_0, x_1)$ solves the entropy-regularized optimal transport problem with cost $||x_0 - x_1||$ and regularization $\lambda = 2\sigma^2$

\item $p_t(x|z) = \CN(x | tx_1 + (1 - t)x_0, t(1-t)\sigma^2$ \textcolor{red}{(Brownian bridge between $x_0$ and $x_1$)}

\item $u_t(x|z) = \frac{1 - 2t}{2t(1-t)}\left(x - (tx_1 + (1 - t) x_0) \right) + (x_1 - x_0)$
\end{enumerate}

The ability to establish correlations through reasonably sized pilot datasets may be enhanced through the use of flow matching. 
In cases where the high-fidelity model can only be evaluated a limited number of times, the flow matching surrogate can enable a fast mapping from the particles propagated under a lower-fidelity model likelihood to those under the highest fidelity likelihood. 
Moreover, it can be parametrized to use particles based on multiple lower-fidelity models for even greater savings.

\section{Methodology}

We begin with the assumption that we have access to a black-box model ensemble $f_i(\theta)$, $i=1, \cdots, M$ where the subscript indexes models from the highest to the lowest fidelity, and $\theta \in \RR^{n_\theta}$ is a random vector.

% Let $y_0$ and $y_1$ be observations 


% \begin{equation}
%     y_i^{(j)} = f_i(\theta)^{(j)} + \varepsilon_i
%     \label{eq: dgp_fid}
% \end{equation}

% where \varepsilon is homoskedastic noise associated with  

Then, the log-likelihood functions corresponding to these outputs can be expressed as:

\begin{align}
 L_i = \log p(f_i | \theta)
 \label{eq: lik_fid}
\end{align}

The score function for the target density then simplifies to the following, where the normalizing constant vanishes because it is independent of the model parameters:

\begin{align}
    \nabla_{\theta} \log p_i(\theta \mid D) = L_i \log(p(\theta)),
    \label{eqn: bayes_rule_score}
\end{align}

\subsection{Baseline: Bifidelity SVGD}


% Then assuming an additive correction, a simple Stein update for the parameters $\theta_i$ would become:

Before we propose the multi-fidelity construction It is is still instructive to demonstrate the basic idea of partitioning the likelihood evaluation costs through a bifidelity surrogate SVGD construction, where a subset of particles evaluated under the lower-fidelity target density are `corrected' through the corresponding high-fidelity density gradient update.

Let $n_0$ be the number of particles that are updated under both high and low-fidelity likelihood models. Let $n_1$ represent the number of particles updated only under the low-fidelity representation.

Then the new updates take the form:
\begin{align}
    x_i^{l + 1} &\leftarrow x_i^l + \epsilon_l\hat{\phi}^{\ast}_{\text{BF}}(x_i^l)
    \label{eq: bf_svgd_basic}
\end{align}

where:
\begin{align}
    \hat{\phi}^{\ast}_{\text{BF}}(x) &= \frac{1}{n_1}\left[\sum_{j=1}^{n_1} k(x_j^l, x) \nabla_{x_j^l}\log p_1(x_j^l) + \nabla_{x_j^l}k(x_j^l, x)\right]
    + \frac{1}{n_0}\left[\sum_{j=1}^{n_0} k(x_j^l, x) \nabla_{x_j^l}\log \left(\frac{p_0(x_j^l)}{p_1(x_j^l)}\right)\right]
    \label{eq:bf_gradient_expr}
\end{align}

The additive correction can be learnt either for a single subset of initial particles that does not change during the transport process or subsets that are randomized from iteration to iteration. In this case, the repulsion term is driven by particle positions under the misspecified likelihood during the initial step and a mixture of particles driven by multiple fidelities in subsequent steps.

We show an example of mapping from an initial set of particles to a bimodal high-fidelity target density, leveraging a lower-fidelity density which corresponds to one of the modes:


To evaluate the efficacy of the bifidelity update, we first visualize the evolution of the contributors to the velocity field i.e. SVGD gradient at different time steps.









% \begin{equation}
%     \theta_i \leftarrow \theta_i + \epsilon \phi^{\ast}(\theta)
% \end{equation}

% where 

% \begin{equation}
%     \phi^{\ast}(\theta) = \frac{1}{M}\sum_{j=1}^{M}[k(\theta_j, \theta) \grad_{\theta_j}[(L_{1}(\theta_j) + L_{\Delta}(\theta_j) + \log p(\theta)] + \grad_{\theta_j} k(\theta_j, \theta)]
% \end{equation}

% While this framework does on the surface, construct an update equation based on multiple models, it comes with its fair share of drawbacks: for one, it requires us to propose an adequate estimator for $L_{\Delta}$, followed by a design for pilot samples to correlate the two models and a formal accounting of the bias introduced by this correction. 
% A viable strategy is to allocate model fidelities to individual particles, and construct an unbiased estimator of the gradient flow to target cost savings via limited and strategic querying of high-fidelity model / likelihood updates. 
% While not quite used for Bayesian inference, similar ideas have found traction in the field of rare event estimation \cite{dhulipala_bayesian_2022}, where the objective is to use the uncertainty from low-fidelity model outputs as an indicator of proximity of particles to the failure boundary. Following this, the limit state function can be computed through the higher-fidelity model and sample the boundary efficiently.

\subsection{Generalized Multifidelity SVGD}

We treat the Stein update functions as scalar QoIs. Then the high-fidelity particle positions i.e. those generated using the likelihood of the high-fidelity model are given by:

\begin{equation*}
    \theta^{(0)}_i \leftarrow \theta^{(0)}_i + \epsilon \phi^{\ast}(\theta^{(0)})
\end{equation*}

And similarly, for lower-fidelity models indexed from $1$ to $m$, we can compute the particle positions as:

\begin{equation*}
    \theta^{(l)}_i \leftarrow \theta^{(l)}_i + \epsilon \phi^{\ast}(\theta^{(l)}), \quad l=1, \cdots, m
\end{equation*}

where

\begin{align}
    \phi^{\ast}(\theta^{(0)}) &= \frac{1}{M^{(0)}}\sum_{j=1}^{M^{(0)}}[k(\theta_j, \theta) \grad_{\theta_j}[L_{0} + \log p(\theta_j)] + \grad_{\theta_j} k(\theta_j, \theta)] \\
    &= \frac{1}{M^{(0)}}\sum_{j=1}^{M^{(0)}}[(k(\theta_j, \theta^{(0)})  + k(\theta_j, \theta^{(1)}) + \cdots + \nonumber\\ 
    &k(\theta_j, \theta^{(m)})) \grad_{\theta_j}[L_{0} + \log p(\theta_j)]] + \grad_{\theta_j} [k(\theta_j, \theta^{(0)}) + \cdots + k(\theta_j, \theta^{(m)})]\\ \nonumber
\end{align}

\begin{align}
    \phi^{\ast}(\theta^{(l)}) &= \frac{1}{M^{(l)}}\sum_{j=1}^{M^{(l)}}[k(\theta_j, \theta) \grad_{\theta_j}[L_{l} + \log p(\theta_j)] + \grad_{\theta_j} k(\theta_j, \theta)] \\
    &= \frac{1}{M^{(l)}}\sum_{j=1}^{M^{(l)}}[(k(\theta_j, \theta^{(0)})  + k(\theta_j, \theta^{(1)}) + \cdots + \nonumber \\
    &k(\theta_j, \theta^{(m)})) \grad_{\theta_j}[L_{l} + \log p(\theta_j)] + \log p(\theta_j)]] + \grad_{\theta_j} k(\theta_j, \theta) \nonumber \\ 
    &= \frac{1}{M^{(l)}}\sum_{j=1}^{M^{(l)}}[(k(\theta_j, \theta^{(0)})  + k(\theta_j, \theta^{(1)}) + \cdots + \nonumber\\ 
    &k(\theta_j, \theta^{(m)})) \grad_{\theta_j}[L_{l} + \log p(\theta_j)]] + \grad_{\theta_j} [k(\theta_j, \theta^{(0)}) + \cdots + k(\theta_j, \theta^{(m)})]\\ \nonumber
\end{align}

While the $M$ particles are allocated to different models as $M_0, M_1, \cdots, M_m$, the above updates still use all the particles to compute the map at each iteration - this introduces interactions between the multifidelity particles in \stkout{\emph{both} terms of the update: the first term driving the particles to regions of high-probability} \textcolor{red}{(unfortunately, this is not true. When you cut off the kernel matrix to remove the particle interactions amongst the other fidelities, it turns out that only the particle distances corresponding to that fidelity weigh the likelihood)} the second term of the update accounting for the repulsive force that prevents particle collapse into local modes. 
The repulsive force i.e. the kernel gradient evaluates a shared kernel function over different subsets of $\theta$ in each term, while the force driving particles towards high-probability regions weighs the kernel function by the individual log-likelihoods.
\textcolor{red}{Interestingly, another algorithmic choice here is the choice of particle positions to use in a given fidelity's updates. For example, if we updated the low-fidelity particle's positions first, these updates could be transmitted to and used by the high fidelity update mechanism in the same step instead of both updates using previously computed positions.}

The above formulation continues to allow flexibility with regards to the specification and the number of lower-fidelity model; but it also raises additional open questions regarding the following aspects:

\begin{enumerate}
    \item The choice of the kernel function for individual model fidelities

    \item The allocation of particles to different model fidelities (disjoint / overlapping) and how it changes as we step through the method.

    \item The number and allocation of particles used in computations of the kernel function i.e. neighbourhood selection for particles.

    \item The frequency of updates to particles of different fidelities, and the update sharing frequency within a single iteration of the outer loop.

    \item The stopping criterion that assesses convergence to the target distribution.

    \item Quantitative measurements of discrepancy between distributions fitted via single-fidelity and multiple-fidelity methods.
\end{enumerate}

% \subsection{KL-Divergence Based Correction with kNN Estimator}





% Perhaps the more pertinent question before we discuss the finer details of the implementation is why we expect this to work at all. Consider the elementary case where we only have access to a single model for likelihood computation. 
% There are two possible bottlenecks: 1. the full data log-likelihood computation is prohibitive on account of the need to process very large-scale data, requiring the modification of standard inference algorithms to handle distributed or streaming data settings 2. the high costs of running the forward model renders the inference intractable without resorting to surrogate approximations. 

% A prominent line of work that deals with the first case uses the so-called `consensus Monte Carlo methods' \citep{rabinovich_variational_2015,scott_bayes_2016}. Here a data sharding (partitioning) step is followed by independent posterior sampling on multiple machines conditioned on the data partitions. 
% The communication overhead is avoided by combining the posterior draws to form a `consensus' belief about the model unknowns.
% The algorithm is exact for Gaussian posteriors, but has been found to be broadly useful for non-Gaussian settings too.

% Rather than modifying the inference algorithm, practitioners have recently focused on exploiting redundancies in the dataset by identifying a weighted subset of the data known as the \emph{coreset}, that originated in computational geometry \citep{agarwal_geometric_2007} and later found popularity in scalable clustering methods.
% Discussions on Bayesian coreset methods \citep{huggins_coresets_2016,campbell_automated_2019} focus on high-probability guarantees for the approximation quality of the resulting log-likelihood and demonstrate their superior performance over random subsampling methods. A superficial analogy we can draw is the treatment of the high-fidelity particles as our coreset, and the kernel function that accounts for all model fidelities as a means to enforce a weighted log-likelihood.

% Akin to coresets, the concept of `graph coarsening' is ubiquitous all across scientific computing and machine learning methods for graph representation \citep{chen_graph_2022}. The goal of graph coarsening is to uncover a smaller graph that faithfully reflects the structure of the original graph. The coarse graph/s is/are typically determined using notions of pairwise similarity, algebraic distance, independent sets, sketching based on leverage scores and other criteria. 
% In this framework, we can treat the selection of high-fidelity particles as a form of graph coarsening, if the features for these particles are constructed from aggregation of lower-fidelity particles rather than assignment of existing particles.

% While these techniques do not directly relate to the workings or performance of our proposed method relying on the non-parametric version of the SVGD method, they serve as useful guiding principles to improve practical implementations which typically rely on random particle subsampling for efficient kernel computations. 

% A simpler analogy arises from the purview of dynamical systems: the asymptotic behaviour of SVGD is characterized by the Vlasov equation \citep{liu_stein_2017}. 
% Then, deploying a multistep predictor-corrector method to solve the resulting ordinary differential equation would be akin to performing a sequence of low-fidelity predictions for particle positions to serve as the initial guess, followed by a correction step for some or all particles via the high-fidelity likelihood i.e. the fidelity assignment \emph{alternates} rather than preceding or following the model updates .
% The benefits of such an approach would be practically realized when the equivalent cost in terms of single-fidelity evaluations is measurably brought down in implementations. 
% However, determination of `when to correct, what to correct' is also a non-trivial problem.

% In the following sections, we will lay out a framework to evaluate some of these key choices and provide more details of our proposed method.

% \subsection{Particle assignment to various fidelities}

% \emph{Method 0: Non-randomized Assignment}
% The simplest possible setting does not attempt to reassign particles, instead, at every iteration, a fixed proportion of the particles is treated as the lower-fidelity likelihood update while the remaining are high-fidelity updates. 
% We look at the number of iterations required for convergence by monitoring the hyperparameters of the kernel function.

% \noindent \emph{Method 1: Randomized Assignment}
% To correct for possible bias due to incorrect particle assignment, randomized assignment shuffles the particles between the different fidelities after one or more update steps.

% \noindent \emph{Method 2: Assignment based on clustering methods}
% For the first time, we consider the setting where a small subset of particles is deemed to be non-redundant and a summarization of the dataset. Here, we incorporate a clustering-based heuristic that determines the high-fidelity particles based on a suitable measure of distance.
% It is of particular importance to ensure that the cost of pairwise distance computations does not grow or outweigh the cost of simpler methods such as randomized assignment i.e. there is possibility of reuse for these when we actually perform the Stein updates.


% \noindent \emph{Method 3: Resource allocation using model correlations}

% \subsection{Graph Network Based Propagation}
% Deep learning methods that model arbitrary relational structures between elements, such as graph networks (GNs), are surveyed in \cite{battaglia_relational_2018}. 

% GNs model a domain by decomposing it into a graphical representation $\CG=\{\CV, \CE\}$ - nodes $\CV$ that represent individual features and edges $\CE$ that describe directed or undirected interactions between nodes. 
% The learning proceeds by first defining a neighbourhood for individual nodes. This is followed by a message-passing block that updates node features based on aggregating interactions with all or a subset of its possible neighbours. Chaining several such blocks with intermediate nonlinear activations results in a GN which accepts an arbitrarily sized graph as input and returns either a transformed graph or some global property of the graph as the output. 

% Superficially translating the graphical representation to a particle-based updates / normalizing flow setup suggests the propagation of augmented node features $[x, \log p(x)]$ where both the particle positions $x$ and their log-likelihoods $\log p(x)$ are updated as they pass through the message-passing blocks. Two key advantages offered by graph-based methods are their inherent invariance to permutations (unlike MLPs) and their ability to operate on graphs of arbitrary sizes without modifications to the architecture. 
% This feature translates to so-called `discretization invariance' in expressive methods for learning PDEs such as neural operators \citep{Li2020} which exhibit good generalization properties even when trained on low-resolution data. GNs have been successfully used to simulate complex physics with long rollout times, e.g. \cite{sanchez-gonzalez_learning_2020}.

% Without additional constraints however, a GN is not directly compatible with the workflow of particle-based methods for inference, where the key goal is to constrain the transport of individual particles by minimizing a particular discrepancy function that characterizes the quality of the posterior approximation. 
% It is also unclear if the discretization-invariance property grants any advantages in better representation of the posterior density in low-data settings. 
% However, the advances in scaling GNs to very large graphs with millions of nodes suggest opportunities to influence the development of scalable particle methods for inference and offer a more flexible workflow. 
% We begin with a brief overview of our proposed changes below:

% \subsection{Particle Swarm Optimization based methods}
% (with a specific interest in multi-population cooperative particle swarm optimization, even though we are not specifically targeting optimization) - see \url{https://ieeexplore.ieee.org/document/9967774} for an example.


\subsection{Sparse Kernel Updates}

\section{Results}

\subsection{Convergence of Single Level MF-SVGD}

\subsection{Convergence of Nested MF-SVGD}

\subsection{Optimal Experimental Design via MF-SVGD Bayesian Inference}


% \subsection{Generalizing Compressed Representations to Higher Particle Counts}
% We consider the case where the original inference is performed on a small set of $N$ particles, but density estimation may be desired with additional particles at test time. 
% Then we need the ability to perform rapid approximation of positions for $K$ new particles introduced at test time. 
% This would require $\Mycomb[(N+K)]{2} - \Mycomb[N]{2} = \frac{1}{2}(K^2 + 2NK - K)$ additional distance calculations and $K$ potentially expensive likelihood computations followed by density estimation in the naive setting. 
% By contrast, a trained parametric simulator that 1. is `discretization invariant' in some sense by capturing the key relationships between existing particles 2. can allocate a large fraction of particles to one or more lower-fidelity models can rapidly convert these to high-quality posterior samples.

\subsection{}

\section{Conclusions and Extensions}

\appendix
\section{An example appendix} 
\lipsum[71]



\section*{Acknowledgments}

\bibliographystyle{siamplain}
\bibliography{local,references}

\end{document}

